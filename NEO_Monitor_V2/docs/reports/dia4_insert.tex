\documentclass[12pt,a4paper]{article}
\usepackage[utf8]{inputenc}
\usepackage[portuguese]{babel}
\usepackage{geometry}
\usepackage{graphicx}
\usepackage{xcolor}
\usepackage{listings}
\usepackage{hyperref}
\usepackage{fancyhdr}

\geometry{margin=2.5cm}

\title{\textbf{NEO Monitor V2}\\Relatório Dia 4: Inserção Manual \& CSV Import}
\author{UBI - Projeto NEO Monitor}
\date{24 Dezembro 2024}

\begin{document}

\maketitle
\tableofcontents
\newpage

\section{Resumo Executivo}

O Dia 4 foi concluído com sucesso total, implementando funcionalidades críticas de inserção de dados:

\begin{itemize}
    \item \textbf{Inserção Manual}: Form completo com 9 campos e validação robusta
    \item \textbf{Importação CSV}: Import threaded com progress bar em tempo real
    \item \textbf{Validação}: Sistema completo de validação de campos
    \item \textbf{UX}: Mensagens personalizadas e feedback visual consistente
    \item \textbf{Progresso}: 4/7 dias (57\% completo)
\end{itemize}

\section{Implementações Principais}

\subsection{Insert Window (insert.py)}

\textbf{Arquitetura:} QTabWidget com 2 tabs

\subsubsection{Tab 1: Inserção Manual}

\textbf{Campos do Form:}
\begin{enumerate}
    \item \textbf{Designação (pdes)} - QLineEdit (obrigatório)
    \begin{itemize}
        \item Identificador único do asteroide
        \item Ex: ``2024 AB'', ``99942''
    \end{itemize}
    
    \item \textbf{Nome Completo} - QLineEdit (obrigatório)
    \begin{itemize}
        \item Nome do asteroide
        \item Ex: ``Apophis'', ``Bennu''
    \end{itemize}
    
    \item \textbf{NEO (Near-Earth Object)} - QCheckBox
    \begin{itemize}
        \item Default: Checked
        \item Flag binária (0 ou 1)
    \end{itemize}
    
    \item \textbf{PHA (Potentially Hazardous)} - QCheckBox
    \begin{itemize}
        \item Default: Unchecked
        \item Flag binária (0 ou 1)
    \end{itemize}
    
    \item \textbf{H Magnitude} - QLineEdit com QDoubleValidator (opcional)
    \begin{itemize}
        \item Magnitude absoluta do asteroide
        \item Aceita valores negativos e decimais
    \end{itemize}
    
    \item \textbf{Diâmetro (km)} - QLineEdit com QDoubleValidator (opcional)
    \begin{itemize}
        \item Validação: Deve ser $>$ 0 se preenchido
    \end{itemize}
    
    \item \textbf{MOID (UA)} - QLineEdit com QDoubleValidator (opcional)
    \begin{itemize}
        \item Minimum Orbit Intersection Distance em Unidades Astronômicas
        \item Validação: Deve ser $\geq$ 0 se preenchido
    \end{itemize}
    
    \item \textbf{MOID (LD)} - QLineEdit com QDoubleValidator (opcional)
    \begin{itemize}
        \item MOID em Distâncias Lunares
        \item Validação: Deve ser $\geq$ 0 se preenchido
    \end{itemize}
    
    \item \textbf{Albedo} - QLineEdit com QDoubleValidator (opcional)
    \begin{itemize}
        \item Refletividade do asteroide
        \item Validação: Deve estar entre 0.0 e 1.0
    \end{itemize}
\end{enumerate}

\textbf{Validações Implementadas:}
\begin{itemize}
    \item Campos obrigatórios não vazios (Designação, Nome)
    \item Validação numérica automática (QDoubleValidator)
    \item Diâmetro: apenas positivo ($>$ 0)
    \item MOID: apenas não-negativo ($\geq$ 0)
    \item Albedo: range válido (0.0 - 1.0)
    \item Mensagens de erro específicas para cada tipo de validação
\end{itemize}

\textbf{Funcionalidades:}
\begin{itemize}
    \item Botão ``Inserir Asteroide'' (azul \#1976d2)
    \item INSERT direto na tabela \texttt{dbo.Asteroide}
    \item Retorna ID gerado via \texttt{SCOPE\_IDENTITY()}
    \item Mensagem de sucesso mostra ID + Designação + Nome
    \item Form auto-limpa após inserção bem-sucedida
    \item Focus automático no primeiro campo (Designação)
\end{itemize}

\subsubsection{Tab 2: Importação CSV}

\textbf{Componentes:}
\begin{enumerate}
    \item \textbf{Botão ``Selecionar Ficheiro CSV''}
    \begin{itemize}
        \item Abre QFileDialog
        \item Filtro: ``Ficheiros CSV (*.csv);;Todos os ficheiros (*.*)''
        \item Mostra nome do ficheiro selecionado
    \end{itemize}
    
    \item \textbf{Progress Bar}
    \begin{itemize}
        \item Atualização dinâmica (0-100\%)
        \item Cor azul (\#1976d2) consistente com a app
        \item Label mostra: ``Importando: X/Y registos (Z\%)''
    \end{itemize}
    
    \item \textbf{Botão ``Importar CSV''}
    \begin{itemize}
        \item Enabled apenas se ficheiro selecionado
        \item Disabled durante importação (evita cliques duplos)
        \item Re-enabled após conclusão
    \end{itemize}
\end{enumerate}

\textbf{Threading Implementation:}
\begin{lstlisting}[language=Python]
class CSVImportThread(QThread):
    progress_update = pyqtSignal(int, int, float)
    import_complete = pyqtSignal(int)
    import_error = pyqtSignal(str)
    
    def run(self):
        def progress_callback(current, total, elapsed):
            self.progress_update.emit(current, total, elapsed)
        
        total_inserted = importar_neo_csv(
            self.conn, self.filepath, progress_callback
        )
        self.import_complete.emit(total_inserted)
\end{lstlisting}

\textbf{Vantagens do Threading:}
\begin{itemize}
    \item UI não bloqueia durante import
    \item Progress bar atualiza em tempo real
    \item Utilizador pode ver progresso (X/Y registos)
    \item Experiência profissional e responsiva
\end{itemize}

\subsection{Integração MainWindow}

\textbf{Modificações em} \texttt{main\_window.py}:
\begin{itemize}
    \item Import: \texttt{from frontend.ui.insert import InsertWindow}
    \item Linha 53: \texttt{self.insert\_page = InsertWindow(self.conn)}
    \item Substituição do placeholder pela página funcional
    \item Footer atualizado: ``v2.0 - Dia 4''
\end{itemize}

\textbf{Navegação:}
\begin{itemize}
    \item Página 4 no \texttt{QStackedWidget}
    \item Botão ``➕ Inserção'' no sidebar
    \item Conexão BD mantida entre navegações
    \item State persistence correto
\end{itemize}

\subsection{Message Utils Enhancement}

\textbf{Nova função adicionada:}
\begin{lstlisting}[language=Python]
def show_message(parent, title: str, message: str, 
                 icon=QMessageBox.Icon.Information):
    # Escolhe cor do botão baseada no tipo de ícone
    if icon == QMessageBox.Icon.Critical:
        button_color = "#d32f2f"  # Vermelho
    elif icon == QMessageBox.Icon.Warning:
        button_color = "#ffa726"  # Laranja
    else:
        button_color = "#1976d2"  # Azul
    
    # QMessageBox com contraste perfeito
\end{lstlisting}

\textbf{Utilização em Insert Window:}
\begin{itemize}
    \item Validação de campos obrigatórios $\rightarrow$ Warning (laranja)
    \item Validação de rangos $\rightarrow$ Warning (laranja)
    \item Sucesso na inserção $\rightarrow$ Information (azul)
    \item Erros de BD $\rightarrow$ Critical (vermelho)
\end{itemize}

\section{Estatísticas}

\begin{tabular}{|l|r|}
\hline
\textbf{Métrica} & \textbf{Valor} \\
\hline
Ficheiros novos criados & 1 \\
Ficheiros modificados & 2 \\
Linhas de código (insert.py) & $\sim$550 \\
Linhas de código (Total Dia 4) & $\sim$600 \\
Campos de form implementados & 9 \\
Validações implementadas & 6 \\
Threads criadas & 1 (CSV Import) \\
Taxa de sucesso testes & 100\% \\
\hline
\end{tabular}

\vspace{1cm}

\textbf{Complexidade:}
\begin{itemize}
    \item \textbf{Inserção Manual}: Média (6/10)
    \item \textbf{CSV Import Threading}: Alta (8/10)
    \item \textbf{Validação}: Média (5/10)
\end{itemize}

\section{Testes Realizados}

\subsection{Testes Funcionais}

\textbf{Inserção Manual (Sucesso):}
\begin{itemize}
    \item [\checkmark] Inserção com campos obrigatórios apenas
    \item [\checkmark] Inserção com todos os campos preenchidos
    \item [\checkmark] Form limpa após sucesso
    \item [\checkmark] ID gerado aparece na mensagem
    \item [\checkmark] Registo visível na Search após inserção
\end{itemize}

\textbf{Validação (Edge Cases):}
\begin{itemize}
    \item [\checkmark] Designação vazia $\rightarrow$ Erro
    \item [\checkmark] Nome vazio $\rightarrow$ Erro
    \item [\checkmark] Texto em campo numérico $\rightarrow$ Bloqueado
    \item [\checkmark] Diâmetro negativo $\rightarrow$ Erro
    \item [\checkmark] MOID negativo $\rightarrow$ Erro
    \item [\checkmark] Albedo = 1.5 $\rightarrow$ Erro (range 0-1)
    \item [\checkmark] Albedo = -0.1 $\rightarrow$ Erro
    \item [\checkmark] Valores válidos $\rightarrow$ Sucesso
\end{itemize}

\textbf{CSV Import:}
\begin{itemize}
    \item [\checkmark] File dialog abre corretamente
    \item [\checkmark] Nome do ficheiro aparece após seleção
    \item [\checkmark] Botão ``Importar'' enabled apenas se ficheiro selecionado
    \item [\checkmark] Progress bar atualiza dinamicamente
    \item [\checkmark] Label mostra progresso (X/Y registos)
    \item [\checkmark] UI não bloqueia durante import
    \item [\checkmark] Mensagem final com total inserido
    \item [\checkmark] Dashboard atualiza após import
\end{itemize}

\textbf{Navegação:}
\begin{itemize}
    \item [\checkmark] Sidebar ``➕ Inserção'' funcional
    \item [\checkmark] Navegação para outras páginas mantém BD conectada
    \item [\checkmark] Voltar para Inserção não mantém dados anteriores
\end{itemize}

\subsection{Testes de Contraste}

Todos os elementos validados para legibilidade (WCAG AA):
\begin{itemize}
    \item Títulos: \#212121 (preto - contraste máximo)
    \item Labels: \#212121 (preto)
    \item Subtítulos: \#424242 (cinza escuro)
    \item Placeholders: \#757575 (cinza médio - visível)
    \item Botão principal: Branco em \#1976d2 (azul)
    \item Progress bar: \#1976d2 com texto \#212121
\end{itemize}

\textbf{Resultado:} Zero problemas de legibilidade.

\section{Decisões Técnicas}

\subsection{Por que não usar \texttt{manual\_insert\_full\_record}?}

A função existente no backend requer contexto completo:
\begin{itemize}
    \item Centro de Observação (código, nome, país)
    \item Equipamento (nome, tipo, modelo)
    \item Astrónomo (nome completo)
    \item Software (nome, versão)
    \item Observação (data, modo, notas)
\end{itemize}

\textbf{Problema:} Para adicionar um asteroide simples, obrigaria 5 forms adicionais.

\textbf{Solução:} INSERT direto na tabela \texttt{Asteroide} com apenas os campos essenciais.

\textbf{Vantagem:} UX mais simples e rápida para caso de uso comum.

\subsection{Threading vs Async}

\textbf{Escolha:} QThread

\textbf{Razões:}
\begin{itemize}
    \item PyQt6 não tem async/await nativo
    \item QThread é thread-safe e bem integrado
    \item Signals permitem comunicação UI $\leftrightarrow$ Thread
    \item Solução robusta e testada
\end{itemize}

\subsection{Validação Client-Side}

\textbf{Abordagem em camadas:}
\begin{enumerate}
    \item \textbf{QDoubleValidator}: Bloqueia entrada não-numérica
    \item \textbf{Lógica Python}: Valida ranges antes de INSERT
    \item \textbf{Database}: Constraints SQL (fallback)
\end{enumerate}

\textbf{Vantagem:} Feedback imediato ao utilizador, sem trip à BD.

\section{Próximos Passos (Dia 5)}

\begin{enumerate}
    \item \textbf{Alertas Screen}
    \begin{itemize}
        \item Tabela de alertas da BD
        \item Cores por nível (verde/amarelo/vermelho)
        \item Filtro: Ativos/Todos
    \end{itemize}
    
    \item \textbf{Monitorização Screen}
    \begin{itemize}
        \item Tab 1: Ranking PHAs (mais perigosos)
        \item Tab 2: Centros de Observação (estatísticas)
        \item Tab 3: Aproximações Críticas (próximas da Terra)
    \end{itemize}
\end{enumerate}

\section{Conclusão}

O Dia 4 foi concluído com sucesso total. A funcionalidade de inserção está completa, robusta e profissional:

\begin{itemize}
    \item[\checkmark] Form manual intuitivo com validação completa
    \item[\checkmark] CSV import otimizado com feedback visual
    \item[\checkmark] Threading evita bloqueio da UI
    \item[\checkmark] Mensagens personalizadas por tipo de evento
    \item[\checkmark] Contraste perfeito em todos os elementos
    \item[\checkmark] Zero bugs encontrados
\end{itemize}

\textbf{Status:} Projeto 57\% completo (4/7 dias)\\
\textbf{Deadline:} No prazo (31 Dezembro)

\vspace{1cm}

\textbf{Próxima Sessão:} Dia 5 - Alertas \& Monitorização (28 Dezembro 2024)

\end{document}
